\documentclass[10pt]{article}
\usepackage{fontspec}
\setmainfont[Ligatures=TeX]{Didot}
\usepackage[utf8]{inputenc}
\usepackage[papersize={17in, 11in}]{geometry}
\usepackage[absolute]{textpos}
\TPGrid[0.5in, 0.25in]{23}{24}
\usepackage{palatino}
\parindent=0pt
\parskip=12pt
\usepackage{nopageno}
\usepackage{graphicx}
\graphicspath{ {./images/} }
\usepackage{lilyglyphs}
\usepackage{amsmath}
\usepackage{tikz}
\newcommand*\circled[1]{\tikz[baseline=(char.base)]{
            \node[shape=circle,draw,inner sep=1pt] (char) {#1};}}
\begin{document}

\begin{textblock}{23}(2.667, 1)
\huge FOREWORD
\end{textblock}

\begin{textblock}{23}(10.5, 1)
\huge PREFACIO
\end{textblock}

\begin{textblock}{23}(18.333, 1)
\huge VORWORT
\end{textblock}

\begin{textblock}{7.333}(0, 2)
$\acute{o} \gamma \kappa o \varsigma$ (Onkos) is a Greek word that means ``volume," or ``mass," and has come to also mean ``tumor." This word is the source of the derivation of such words as $Oncology$: the study and treatment of tumors.  \\
\phantom{text} \hfill (G.R.E.)
  \end{textblock}





\begin{textblock}{7.333}(7.8333, 2) 
En Espa\~nol... \\
\phantom{text}  \hfill (Tr : Juanita Pineda)
 \end{textblock}





\begin{textblock}{7.333}(15.666, 2) 
$\acute{o} \gamma \kappa o \varsigma$ (Onkos) ist ein griechisches Wort, das \guillemotleft Volumen\guillemotright \ oder \guillemotleft Masse\guillemotright \ bedeutet und zu gekommen ist bedeutet auch \guillemotleft Tumor\guillemotright. Dieses Wort ist die Quelle der Ableitung solcher W\"orter wie $Onkologie$: Untersuchung und Behandlung von Tumoren. \\
\phantom{text} \hfill (Übs: G.R.E.)
 \end{textblock}

\begin{textblock}{23}(1.167, 5)
\huge PERFORMANCE NOTES
\end{textblock}

\begin{textblock}{23}(9, 5)
\huge NOTAS DE RENDIMIENTO
\end{textblock}

\begin{textblock}{23}(16.833, 5)
\huge LEISTUNGSHINWEISE
\end{textblock}

\begin{textblock}{7.333}(0, 6)
\pmb{Microtones}:
\end{textblock}

\begin{textblock}{23}(0, 6.5)
\includegraphics[width=0.28\textwidth]{microtones.png}
\end{textblock}

\begin{textblock}{7.333}(0, 8.5)
\pmb{Bow Angle Indications} : The upper bracket above the staff is filled with information regarding the angle at which the bow is to be held such as $col \ legno \ tratto$ (abbreviated as $clt.$), $1/2 \ col \ legno \ tratto$ (abbreviated as $1/2 \ clt.$), $1/2 \ hair \ arco,$ $3/4 \ hair \ arco,$ and $flat \ hair \ arco.$ These angles should be maintained at a constant level for the duration of the bracket-demarcated passage. When this bracket is not present, the performer should default to $ordinario$ bowing techniques.

\pmb{String Position Indications} : The indications of string positions such as $sul \ tasto$ (abbreviated as $st.$), $sul \ ponticello$ (abbreviated as $sp.$), $molto \ sul \ ponticello$ (abbreviated as $msp.$), etc. should be considered as points along the continuum of the string. The performer should make an effort to smoothly transition from one position to the next throughout the duration of the passage covered by the arrow-demarcated dashed line. When this arrow is not present, the performer should default to $ordinario$ bowing techniques.

\pmb{Bow Position Indications} : In various passages throughout this piece, there is notation which represents the horizontal contact point at which the bow touches the string. These positions are written as fractions where \( \frac{1}{7} \) and  \( \frac{1}{5} \) represent $au \ talon$ and \( \frac{7}{7} \) and \( \frac{5}{5} \) represent $punta \ d'arco$.

\pmb{Dynamic Indications} : Throughout this piece, dynamic marks should be considered ``effort dynamics.'' As such, $forte$ represents a heavy bow pressure rather than a ``loud'' resultant sound. Likewise, $piano$ represents a light bow pressure as opposed to a ``quiet'' resultant sound. These indications will often result in unusual bowing timbres when combined with the bow position indications, the string position indications, and finger pressure alterations. These are the desired effects.

\pmb{Miscellaneous} : \circled{1} Tremolos should be performed as fast as possible and not as a measured subdivision of the duration to which it is attached. \circled{2} Diamond note heads represent a left hand finger pressure of a natural harmonic, \circled{3} while a triangle note head indicates a finger pressure in between harmonic and $normale$. \circled{4} Accidentals apply only to the pitch which they immediately precede, but persist through ties.
\end{textblock}










\begin{textblock}{7.333}(7.8333, 6)
\pmb{Microtonos}:
\end{textblock}

\begin{textblock}{23}(7.8333, 6.5)
\includegraphics[width=0.28\textwidth]{microtones.png}
\end{textblock}

\begin{textblock}{7.333}(7.8333, 8.5)
\pmb{Bow Angle Indications} : ...

\pmb{String Position Indications} : ...

\pmb{Bow Position Indications} : ...

\pmb{Dynamic Indications} : ...
\end{textblock}











\begin{textblock}{7.333}(15.666, 6)
\pmb{Mikrotonalen Intervallen}:
\end{textblock}

\begin{textblock}{23}(15.666, 6.5)
\includegraphics[width=0.28\textwidth]{microtones.png}
\end{textblock}

\begin{textblock}{7.333}(15.666, 8.5)
\pmb{Bogenwinkelmarkierungen} : Die obere Klammer \"uber dem Stab ist mit Informationen \"uber den Winkel, unter dem der Bogen gehalten werden soll, wie $col \ legno \ tratto$ (abgek\"urzt als $clt.$), 1/2 $col \ legno \ tratto$ (abgek\"urzt als $1/2 \ clt.$), $1/2 \ Haar \ Arco$, $3/4 \ Haar \ Arco$ und $Flachhaar \ Arco$ gef\"ullen. Diese Winkel sollten f\"ur die Dauer der durch Klammern abgegrenzten Passage konstant gehalten werden. Wenn diese Klammer nicht vorhanden ist, sollte der Bratschist standardm\"a{\ss}ig $ordinario$ Bogetechniken verwenden.

\pmb{Zeichenfolgenpositionen} : Die Angaben zu Zeichenfolgenpositionen wie $sul \ tasto$ (abgek\"urzt als $st.$), $sul \ ponticello$ (abgek\"urzt als $sp.$), $molto \ sul \ ponticello$ (abgek\"urzt als $msp.$) Usw. sollten als Punkte entlang der Kontinuum der Zeichenfolge. Der Bratschist sollte sich bem\"uhen, w\"ahrend der Dauer der Passage, die von der durch einen Pfeil abgegrenzten gestrichelten Linie zur\"uckgelegt wird, einen reibungslosen \"Ubergang von einer Position zur n\"achsten zu erreichen. Wenn dieser Pfeil nicht vorhanden ist, sollte der Bratschist die ordinario Bogungstechniken verwenden.

\pmb{Bogenpositionen} : Die unterste Klammer \"uber dem Stab stellt den horizontalen Kontaktpunkt dar, an dem der Bogen die Saite ber\"uhrt. Diese Positionen werden als Br\"uche geschrieben, wobei \( \frac{1}{7} \) und \( \frac{1}{5} \) $au \ talon$ und \( \frac{7}{7} \) und \( \frac{5}{5} \) $punta \ d'arco$ bedeuten.

\pmb{Dynamische Indikationen} : W\"ahrend des gesamten St\"ucks sollten dynamische Markierungen als \guillemotleft Anstrengungsdynamik\guillemotright \ betrachtet werden. Daher repr\"asentiert forte einen starken Bugdruck und nicht einen \guillemotleft lauten\guillemotright \ resultierenden Klang. Ebenso stellt Piano einen leichten Bogendruck dar, im Gegensatz zu einem \guillemotleft leisen\guillemotright \ resultierenden Klang. Diese Anzeigen f\"uhren h\"aufig zu ungew\"ohnlichen Bogent\"onen, wenn sie mit den Anzeigen der Bogenposition, den Zeichenfolgenpositionen und den \"Anderungen des Fingerdrucks kombiniert werden. Dies sind die gew�nschten Effekte.

\pmb{Verschiedenes} : \circled{1} Tremolos sollten so schnell wie m\"oglich durchgef\"uhrt werden und nicht als gemessene Unterteilung der Dauer, an die sie angef\"ugt sind. \circled{2} Diamant-Notenk\"opfe repr\"asentieren den Fingerdruck einer nat\"urlichen Harmonischen mit der linken Hand, \circled{3} w\"ahrend ein Dreieckskopf einen Fingerdruck zwischen einer nat\"urlichen Harmonischen und Normal anzeigt. \circled{4} Die Vorzeichen beziehen sich nur auf das Tonh\"ohe, dem sie unmittelbar vorangehen, bestehen jedoch durch Krawatten.
\end{textblock}


\begin{textblock}{7.333}(0, 22.2)
c.4'30''
\end{textblock}

\begin{textblock}{7.333}(0, 23)
This piece is dedicated to the memory of Janice Evans and [...] and was composed for Andrew Grishaw as part of the $Viola \ Project$ organzied by Jodi Levitz.
\end{textblock}


\begin{textblock}{7.333}(7.8333, 22.2)
c.4'30''
\end{textblock}

\begin{textblock}{7.333}(7.8333, 23)
...
\end{textblock}


\begin{textblock}{7.333}(15.666, 22.2)
c.4'30''
\end{textblock}

\begin{textblock}{7.333}(15.666, 23)
Dieses St\"uck ist der Erinnerung an Janice Evans und [...] gewidmet und wurde f\"ur Andrew Grishaw im Rahmen des das Bratsche Project von Jodi Levitz komponiert.
\end{textblock}

\end{document}