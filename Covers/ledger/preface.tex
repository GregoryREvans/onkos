\documentclass[10pt]{article}
\usepackage{fontspec}
\setmainfont[Ligatures=TeX]{Didot}
\usepackage[utf8]{inputenc}
\usepackage[papersize={17in, 11in}]{geometry}
\usepackage[absolute]{textpos}
\TPGrid[0.5in, 0.25in]{23}{24}
\usepackage{palatino}
\parindent=0pt
\parskip=12pt
\usepackage{nopageno}
\usepackage{graphicx}
\graphicspath{ {./images/} }
\usepackage{lilyglyphs}
\usepackage{amsmath}
\usepackage{tikz}
\newcommand*\circled[1]{\tikz[baseline=(char.base)]{
            \node[shape=circle,draw,inner sep=1pt] (char) {#1};}}
\begin{document}

\begin{center}
\huge FOREWORD
\end{center}

\begingroup
\begin{center}
\leftskip2.7in
\hspace{10mm} $\acute{o} \gamma \kappa o \varsigma$ (Onkos) is a Greek word that means ``volume," or ``mass," and has come to also mean ``tumor." This word is the source of the derivation of such words as $Oncology$: the study and treatment of tumors.
\rightskip\leftskip
\phantom{text} \hfill (G.R.E.)
\end{center}
\endgroup 

%\vspace*{0.5\baselineskip}

\begin{center}
\huge PERFORMANCE NOTES
\end{center}


\begin{center}
\pmb{Microtones}:
\end{center}

\begin{center}
\includegraphics[width=0.30\textwidth]{microtones.png}
\end{center}

\begingroup
\begin{center}
\leftskip1.4in
\pmb{Bow Angle Indications} : The upper bracket above the staff is filled with information regarding the angle at which the bow is to be held such as $col \ legno \ tratto$ (abbreviated as $clt.$), $1/2 \ col \ legno \ tratto$ (abbreviated as $1/2 \ clt.$), $1/2 \ hair,$ $3/4 \ hair,$ and $flat \ hair.$ These angles should be maintained at a constant level for the duration of the bracket-demarcated passage. When this bracket is not present, the performer should default to $ordinario$ bowing techniques.
\rightskip\leftskip
\phantom{text} \hfill \phantom{()}

\leftskip1.4in
\pmb{String Contact Points} : The indications of string contact positions such as $sul \ tasto$ (abbreviated as $st.$), $sul \ ponticello$ (abbreviated as $sp.$), $molto \ sul \ ponticello$ (abbreviated as $msp.$), etc. should be considered as points along the continuum of the length string. The performer should make an effort to smoothly transition from one position to the next throughout the duration of the passage covered by the arrow-demarcated dashed line. When this arrow is not present, the performer should default to an $ordinario$ position.
\rightskip\leftskip
\phantom{text} \hfill \phantom{()}

\leftskip1.4in
\pmb{Bow Contact Point} : In various passages throughout this piece, there is notation which represents the point at which the bow is touched. These positions are written as fractions where \( \frac{0}{7} \) and  \( \frac{0}{5} \) represent $au \ talon$ and \( \frac{7}{7} \) and \( \frac{5}{5} \) represent $punta \ d'arco$. Passages without these indications should be bowed at the performer's discretion.
\rightskip\leftskip
\phantom{text} \hfill \phantom{()}

\leftskip1.4in
\pmb{Dynamic Indications} : Dynamics within quotation marks should be considered ``effort dynamics.'' As such, $forte$ represents a heavy bow pressure rather than a ``loud'' resultant sound. Likewise, $piano$ represents a light bow pressure as opposed to a ``quiet'' resultant sound. These indications will often result in unusual bowing timbres when combined with the Bow Contact Points, the String Contact Points, and finger pressure alterations. These are the desired effects. Dynamics without quotations should be performed as usual.
\rightskip\leftskip
\phantom{text} \hfill \phantom{()}

\leftskip1.4in
\pmb{Miscellaneous} : \circled{1} Tremolos should be performed as fast as possible and not as a measured subdivision of the duration to which they are attached. \circled{2} Diamond note heads represent a left hand finger pressure of a natural harmonic, \circled{3} while a triangle note head indicates a finger pressure in between harmonic and $normale$. \circled{4} Glissandi with arrows attached represent a pitched glissando where the finger pressure should also smoothly transition from one indication to the next. \circled{5} Accidentals apply only to the pitch which they immediately precede, but persist through ties.
\rightskip\leftskip
\phantom{text} \hfill \phantom{()}
\end{center}
\endgroup

%\vspace*{0.1\baselineskip}


\begin{center}
c.4'30''
\end{center}

%\vspace*{0.1\baselineskip}

\begin{center}
This piece is dedicated to the memory of Janice Evans and Rosa Mar\'ia P\'erez de Cervantes and was composed for Andrew Grishaw as part of the $Viola \ Project$ organzied by Jodi Levitz.
\end{center}



\end{document}