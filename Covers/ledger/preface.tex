\documentclass[10pt]{article}
\usepackage{fontspec}
\setmainfont[Ligatures=TeX]{Didot}
\usepackage[utf8]{inputenc}
\usepackage[papersize={17in, 11in}]{geometry}
\usepackage[absolute]{textpos}
\TPGrid[0.5in, 0.25in]{23}{24}
\usepackage{palatino}
\parindent=0pt
\parskip=12pt
\usepackage{nopageno}
\usepackage{graphicx}
\graphicspath{ {./images/} }
\usepackage{lilyglyphs}
\usepackage{amsmath}
\usepackage{tikz}
\newcommand*\circled[1]{\tikz[baseline=(char.base)]{
            \node[shape=circle,draw,inner sep=1pt] (char) {#1};}}
\begin{document}

\begin{textblock}{23}(2.667, 1)
\huge FOREWORD
\end{textblock}

\begin{textblock}{23}(10.5, 1)
\huge PREFACIO
\end{textblock}

\begin{textblock}{23}(18.333, 1)
\huge VORWORT
\end{textblock}

\begin{textblock}{7.333}(0, 2)
$\acute{o} \gamma \kappa o \varsigma$ (onkos) is a Greek word that means ``volume," or ``mass," and has come to also mean ``tumor." This word is the source of the derivation of such english words as $oncology$: the study and treatment of tumors.......  \\
\phantom{text} \hfill (G.R.E.)
  \end{textblock}





\begin{textblock}{7.333}(7.8333, 2) 
En Espa\~nol... \\
\phantom{text}  \hfill (Tr : Juanita Pineda)
 \end{textblock}





\begin{textblock}{7.333}(15.666, 2) 
Auf Deutsch... \\
\phantom{text} \hfill (Übs: G.R.E.)
 \end{textblock}

\begin{textblock}{23}(1.167, 6)
\huge PERFORMANCE NOTES
\end{textblock}

\begin{textblock}{23}(9, 6)
\huge NOTAS DE RENDIMIENTO
\end{textblock}

\begin{textblock}{23}(16.833, 6)
\huge LEISTUNGSHINWEISE
\end{textblock}

\begin{textblock}{7.333}(0, 7)
\pmb{Microtones}:
\end{textblock}

\begin{textblock}{23}(0, 7.5)
\includegraphics[width=0.28\textwidth]{microtones.png}
\end{textblock}

\begin{textblock}{7.333}(0, 9.5)
\pmb{Bow Angle Indications} : The upper bracket above the staff is filled with information regarding the angle at which the bow is to be held such as $col \ legno \ tratto$ (abbreviated as $clt.$), $1/2 \ col \ legno \ tratto$ (abbreviated as $1/2 \ clt.$), $1/2 \ hair \ arco,$ $3/4 \ hair \ arco,$ and $flat \ hair \ arco.$ These angles should be maintained at a constant level for the duration of the bracket-demarcated passage. When this bracket is not present, the performer should default to $ordinario$ bowing techniques.

\pmb{String Position Indications} : The indications of string positions such as $sul \ tasto$ (abbreviated as $st.$), $sul \ ponticello$ (abbreviated as $sp.$), $molto \ sul \ ponticello$ (abbreviated as $msp.$), etc. should be considered as points along the continuum of the string. The performer should make an effort to smoothly transition from one position to the next throughout the duration of the passage covered by the arrow-demarcated dashed line. When this arrow is not present, the performer should default to $ordinario$ bowing techniques.

\pmb{Bow Position Indications} : In various passages throughout this piece, there is notation which represents the horizontal contact point at which the bow touches the string. These positions are written as fractions where \( \frac{1}{7} \) and  \( \frac{1}{5} \) represent $au \ talon$ and \( \frac{7}{7} \) and \( \frac{5}{5} \) represent $punta \ d'arco$.

\pmb{Dynamic Indications} : Throughout this piece, dynamic marks should be considered ``effort dynamics.'' As such, $forte$ represents a heavy bow pressure rather than a ``loud'' resultant sound. Likewise, $piano$ represents a light bow pressure as opposed to a ``quiet'' resultant sound. These indications will often result in unusual bowing timbres when combined with the bow position indications, the string position indications, and finger pressure alterations. These are the desired effects.

\pmb{Miscellaneous} : \circled{1} Tremolos should be performed as fast as possible and not as a measured subdivision of the duration to which it is attached. \circled{2} Diamond note heads represent a left hand finger pressure of a natural harmonic, \circled{3} while a triangle note head indicates a finger pressure in between harmonic and $normale$. \circled{4} Accidentals apply only to the pitch which they immediately precede, but persist through ties.
\end{textblock}










\begin{textblock}{7.333}(7.8333, 7)
\pmb{Microtonos}:
\end{textblock}

\begin{textblock}{23}(7.8333, 7.5)
\includegraphics[width=0.28\textwidth]{microtones.png}
\end{textblock}

\begin{textblock}{7.333}(7.8333, 9.5)
\pmb{Bow Angle Indications} : ...

\pmb{String Position Indications} : ...

\pmb{Bow Position Indications} : ...

\pmb{Dynamic Indications} : ...
\end{textblock}











\begin{textblock}{7.333}(15.666, 7)
\pmb{Mikrotonalen Intervallen}:
\end{textblock}

\begin{textblock}{23}(15.666, 7.5)
\includegraphics[width=0.28\textwidth]{microtones.png}
\end{textblock}

\begin{textblock}{7.333}(15.666, 9.5)
\pmb{Bow Angle Indications} : ...

\pmb{String Position Indications} : ...

\pmb{Bow Position Indications} : ...

\pmb{Dynamic Indications} : ...
\end{textblock}


\begin{textblock}{7.333}(0, 22.5)
c.4'
\end{textblock}

\begin{textblock}{7.333}(0, 23)
This piece is dedicated to the memory of Janice Evans and [...] and was composed for Andrew Grishaw as part of the $Viola \ Project$ organzied by Jodi Levitz.
\end{textblock}


\begin{textblock}{7.333}(7.8333, 22.5)
c.4'
\end{textblock}

\begin{textblock}{7.333}(7.8333, 23)
...
\end{textblock}


\begin{textblock}{7.333}(15.666, 22.5)
c.4'
\end{textblock}

\begin{textblock}{7.333}(15.666, 23)
...
\end{textblock}

\end{document}